\documentclass[11pt]{article}

% Font and color stuff for text
\usepackage{xcolor}
\usepackage{lmodern}
\usepackage[T1]{fontenc}

% Paragraph formatting
\setlength{\parindent}{0pt} 
\setlength{\parskip}{0.125cm} 

% Margins
\topmargin=-0.45in
\evensidemargin=0in
\oddsidemargin=0in
\textwidth=6.5in
\textheight=9.0in
\headsep=0.25in

% Checkboxes
\usepackage{enumitem,amssymb}
\newlist{todolist}{itemize}{2}
\setlist[todolist]{label=$\square$}

% A macro used to make the What I Know Section look "good"
\newcommand{\iknow}[2]{\par\textbf{#1} \textit{#2}}

\title{A Short Report About What I Know}
\author{ Charmaine Aguilar }
\date{\today}

\begin{document}
\maketitle	

\section*{How to complete this assignment}
\begin{todolist}
    \item On a debian 10 droplet:
    \begin{verbatim}
        apt update
        apt install git
        apt install texlive-full
    \end{verbatim}
    \item If the droplet is new, you'll need to configure git again.
    \item Fork the repo
    \item Clone your fork to your machine ( use https or ssh, whichever one you like better )
    \item if the droplet is new and you want to use ssh, you'll need to generate a new key pair
    \item In the repo, create a directory called YourName
    \item Copy the MyReport.tex file into your directory.
    \item Use vim to fill out the "to do"s in the "What I Know" section.
    \item Change "Author" to your name.
    \item Compile using
    \begin{verbatim}
        pdflatex MyReport.tex
    \end{verbatim}
    \item If all goes well, you will see a "MyReport.pdf" in the directory!
    \item \textcolor{red}{ git add MyReport.tex}
    \item {\Large\textcolor{red}{ DO NOT git add -A or git add MyReport.pdf!}}
    \item {\LARGE\textcolor{red}{ DO NOT git add -A or git add MyReport.pdf!}}
    \item I don't want your pdf going onto github, only your .tex  file
    \item Make a pull request
    \item Use sftp to get your pdf to your home computer
    \item Submit the pdf on blackboard
    \item 50 pts for successful PR
    \item 50 pts for PDF
\end{todolist}

\section*{What I Know}

\noindent\textit{Below is a list of a few commands I know, along with brief descriptions of what they do.}
\iknow{cp}{This command is for copying files}
\iknow{cp -r}{This command is to copy directories}
\iknow{mv}{This renames a file and/or directory}
\iknow{rm}{This deletes a file}
\iknow{rm -r}{This deletes a directory and files inside it}
\iknow{ls -a}{This shows a list of files and directories inclusing hidden ones}
\iknow{ls -al}{This shows detailed list of files and directories including hidden files, also displays number of lines it returns}
\iknow{ls -l}{Show a detailed list of files and directories, NOT including hidden files, also displays number of lines it returns}
\iknow{ssh}{A command that lets user connect to a remote computer or your droplet}
\iknow{ssh-keygen}{Generates a new pair of ssh key}
\iknow{cd}{This command allows to change the current directory into a new one}
\iknow{cd ../}{Change directory one step back to previous directory}
\iknow{cd \textasciitilde}{Change directory into root or home}
\iknow{echo}{This command simply displays a line of string literals}
\iknow{cat}{This command concatenates contents of different files together, we also use it to display the contents of a file}
\iknow{adduser}{Adds a new user}
\iknow{deluser}{Delete an existing user}
\iknow{chmod}{Modifies the permissions of the file}
\iknow{chown}{Changes the user and/or group owner of a file or directory}

\section*{Conclusion}
Now you know a tiny bit about using \LaTeX to prepare documents! One thing cool about \LaTeX is that you make beautiful math equations

\begin{equation}
\int_{0}^{\pi}x^2\,dx
\end{equation}

and there are ways to put images in your documents, and code snippets, and you can produce really beautifully formatted PDFs. The lecture notes for this class were all made on an Ubuntu laptop, with vim, in the terminal. And I daresay they are way more beautiful that what could have been easily made with Microsoft Word.

\end{document}
